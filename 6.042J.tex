\documentclass{article}
\usepackage[utf8]{inputenc}
\usepackage{amsmath}
\usepackage{amsthm}
\usepackage{amssymb}
\usepackage{natbib}
\usepackage{graphicx}
\usepackage[utf8]{inputenc}
\usepackage[english]{babel}
\usepackage{chngcntr}

\counterwithin*{equation}{section}
\counterwithin*{equation}{subsection}
\usepackage{hyperref}
\hypersetup{
    colorlinks=true,
    linkcolor=blue,
    filecolor=magenta,      
    urlcolor=cyan,
}
\title{Solution Manual for Problem Sets of MIT OCW 6.042 Fall 2010}
\author{Lu YuXun}
\date{February 2017}


\linespread{1.5}

\newtheorem{theorem}{Theorem}

\begin{document}
\maketitle
\section{About This Solution Manual}
This solution manual is written by Yuxun LU from NAIST. All the answers are done by myself. Although I tried my best to ensure the accuracy of every answer, it is unavoidable that there are some mistakes in this solution manual. If the reader found any error in this solution manual, please do not be hesitate to contact me by leaving the comment on my GitHub/Blog.

This solution manual should be referred only if you've tried solving the problems after many efforts but still didn't find the answer. During solving these problems, I found many of them are valuable for practising the skills of both proof and  arithmetic. \textbf{Spending many HOURS on one problem is not an acceptable reason for referring to the solution manual, spending many DAYS is}. At last, enjoy your problems solving and may you all get success in 6.042 self studying!
\section{Problem Set 1}
\end{document}